
\documentclass[12pt]{scrartcl}
 
\usepackage{graphicx}
\usepackage[english,german]{babel}
 
\begin{document}

\begin{titlepage}
	\centering
	{\scshape\LARGE Ostfalia Hochschule \par}
	\vspace{1cm}
	{\scshape\Large EKdI-Projekt\par}
	\vspace{1.5cm}
	{\huge\bfseries Essay\par}
	\vspace{2cm}
	{\Large\itshape 70476503\par}
	\vfill
	supervised by\par
	Prof. \textsc{Jensen}

	\vfill

% Bottom of the page
	{\large \today\par}
\end{titlepage}



\section{Informatik ist ...}

\begin{document}
\footnotetext[1]{ [WII]}
\footnotetext[2]{ vgl. [WII]}
\footnotetext[3]{ vgl. [52Std], S. 2}
\footnotetext[4]{ [GIS]}
\footnotetext[5]{ vgl. [GET]}
\footnotetext[6]{ vgl. [QUA]}

Die Informatik ist ein vielseitig benutztes und diskutiertes Thema. Es gibt viele verschiedene Bereiche, in denen die Informatik ihren Nutzen hat und sie brachte in diesen Bereichen verschiedenste Innovationen hervor. Doch wie lässt sich Informatik nun erklären? Oder einfacher: Was genau ist Informatik?
\newline
\newline
Um diese Frage zu beantworten, sollte man sich erstmal die gegenteilige Frage stellen: \glqq Was ist nicht Informatik? \grqq\footnotemark[1] Für die meisten Menschen ist Informatik genau das, was es eigentlich nicht ist: nämlich \glqq irgendwas mit Computer \grqq\footnotemark[1]{}.\footnotemark[2]{} Genauer beschrieben Computer zusammenbauen und reparieren, Windows installieren oder dabei helfen Programme oder andere Kleinigkeiten richtig zu bedienen.\footnotemark[2]{}
Denn eigentlich ist Informatik einer der vielfältigsten omnipräsenten Wissenschaften, sie ist überall zu finden, ob im Berufsleben oder privat.\footnotemark[3]{} Mittlerweile sollte sogar davon gesprochen werden, wenn es um das Thema Kultur geht, schließlich prägt die Informatik das Leben so stark wie keine andere Wissenschaft, denn jeder hat ein grundlegendes Verständnis über die Jahre erworben.\footnotemark[3]{} \newline
Logisch gesehen gehört die Informatik zu den Strukturwissenschaften, denn sie versucht \glqq selbstgeschaffene abstrakte Strukturen \grqq\footnotemark[4]{} logisch auf verschiedenen Automaten und Theorien anzuwenden.\footnotemark[5]{} \newline
Physikalisch gesehen ist Informatik eine Naturwissenschaft, zwar beschäftigt sich diese nicht mit der Erforschung von verschiedenen Naturerscheinungen, aber es ist notwendig Kenntnisse der Physik zu haben.\footnotemark[5]{} Um zum Beispiel Quantencomputer zu realisieren, welche in der Informatik Anwendungen und andere Probleme der Informatik beschleunigt.\footnotemark[6]{} \newline
Technisch gesehen ist die Informatik den Ingenieurswissenschaften zuzuordnen, denn ohne die Elektrotechnik und Informatik würde es zum Beispiel in der Automobilindustrie keine technischen Entwicklungen geben.\footnotemark[5]{} \newline
Wenn man die Informatik auf kommunikativer Ebene betrachtet, gehört diese ebenfalls zu den Geisteswissenschaften, schließlich werden die verschiedenen Programmiersprachen als Sprache beschrieben.\footnotemark[5]{} Zwar beschränkt sich diese auf die Befehlsebene im Algorithmus, wodurch die Programmiersprache linguistisch gesehen keine echte Sprache ist.\footnotemark[5]{} Aber trotz dessen ist diese unter Softwareentwicklern als solches zu verstehen, schließlich benötigt man zumeist nichts anderes, als den Programmcode um einen Fehler festzustellen ohne dieses als Programm durchlaufen zu müssen.\footnotemark[5]{}\newpage

\footnotetext[7]{ vgl. [PP], S. 8}
\footnotetext[8]{ vgl. [PP], S. 9}
\footnotetext[9]{ [PP], S. 9}
\footnotetext[10]{ vgl. [PP], S.10}
\footnotetext[11]{ [PP], S.11}
\footnotetext[12]{ vgl. [PP], S.11}
\footnotetext[13]{ [TU], S. 2}
\footnotetext[14]{ vgl. [TU], S. 2}
\footnotetext[15]{ [TU], S. 3}
\footnotetext[16]{ vgl. [TU], S. 3}
\footnotetext[17]{ [PP], S. 5}
Informatik ist ebenfalls als Grundlagenwissenschaft zu verstehen, denn sie ist eine Wissenschaft, welche andere Wissenschaften prägt, welche auch durch die Diskretheit zustande kommt.\footnotemark[7]  \footnotemark[8]{} Außerdem \glqq greift die Informatik bis in die Philosophie hinein \grqq\footnotemark[9]{}, welches durch verschiedene Fragestellungen wie: \glqq Wie verarbeitet der Mensch Informationen? \grqq\footnotemark[9]{}oder ähnliche deutlich wird.\footnotemark[8]{}\newline
Es ist ebenfalls möglich, dadurch das die Informatik sich mit den Konzepten, umsetzen und anwenden in unterschiedlichsten Anwendungsgebieten auseinandersetzt, diese als Ingenieursdisziplin anzusehen.\footnotemark[10]{} Durch diese verschiedenen Anwendungsbereiche sind neue \glqq Wissensbereiche wie Bio-, Geo-, Ingenieur-, Medizin, Medizin-, Rechts-, Verwaltungs-, oder Wirtschaftsinformatik \grqq\footnotemark[11]{} entstanden.\footnotemark[12]{}\newline
Des Weiteren kann man Informatik in verschiedene Teilgebiete einteilen, in die \glqq Theoretische Informatik, (..) Praktische Informatik, (..) Technische Informatik und (..) Angewandte Informatik \grqq\footnotemark[13]{}.\footnotemark[14]{} Theoretische Informatiker achten dabei hauptsächlich auf die Geschwindigkeit und Speicherverbrauch von Algorithmen.\footnotemark[14]{} Ein besonderer Themenabschnitt dieses Teilgebiets ist die Entwicklung künstlicher Intelligenzen.\footnotemark[14]{} \newline 
Um grundsätzliche Lösungskonzepte zu entwickeln, werden praktische Informatiker eingestellt, diese können Computerprogramme entwickeln, welche in großen Softwaresystemen verwendet werden.\footnotemark[14]{} Weitere Anwendungsgebiete der Praktischen Informatik sind zum Beispiel Datenbanken oder \glqq Sicherheit und Bedienungsfreundlichkeit \grqq\footnotemark[13]{}.\footnotemark[14]{}\newline
In der technischen Informatik geht es um die Entwicklung der Hardware für Computersysteme, dabei wird \glqq das Wissen aus Physik, Nachrichtentechnik, Mikroelektronik, Mikro- und Nanotechnik \grqq\footnotemark[15]{} verwendet.\footnotemark[16]{} \newline
Für die Automatisierung von Computerabläufen sind angewandte Informatiker gefragt, diese erschaffen konkrete Anwendungsmöglichkeiten dafür.\footnotemark[16]{} Um verschiedenste Produktionseinheiten zu unterstützen.\footnotemark[16]{}
\newline
\newline
Doch was ist jetzt Informatik? Informatik ist eine Schnittstellenposition und eine eigene Wissenschaft für sich. Informatik ist der \glqq Kern und Motor von Weiterbewegung und Innovation \grqq\footnotemark[17]{} in fast allen Bereichen der Gesellschaft, sie löst Probleme im Bildungssektor, für Unternehmen oder auch privat. Aber vor allem ist Informatik nicht das, was die Mehrheit der Menschen darunter verstehen und dieses Vorurteil sollte so schnell wie möglich, wegen den oben genannten Gründen, von der Gesellschaft entfernen.


 \end{document}

\newpage

\section{Ein erfolgreiches IT-Projekt ...}



 
\end{document}
