\documentclass[12pt]{scrartcl}
 
\usepackage{graphicx}
\usepackage[english,german]{babel}
\usepackage{setspace}
\usepackage{url}
 
\begin{document}

\begin{titlepage}
	\centering
	{\scshape\LARGE Ostfalia Hochschule \par}
	\vspace{1cm}
	{\scshape\Large EKdI-Projekt\par}
	\vspace{1.5cm}
	{\huge\bfseries Essay\par}
	\vspace{2cm}
	{\Large\itshape Marcel Pulst, 70477125\par}
	\vfill
	supervised by\par
	Prof. \textsc{Jensen}

	\vfill

% Bottom of the page
	{\large \today\par}
\end{titlepage}



\section*{Ein erfolgreiches IT-Projekt ...}
\onehalfspacing
…setzt eine gute Teamkommunikation  voraus. Das heißt , dass sich das Team in erster Linie respektiert und sich die Teammitglieder aktiv zuhören, um alle wichtigen Informationen aufzunehmen und die Anliegen der anderen Teammitglieder zu verstehen. Außerdem ist es wichtig, dass das Team Empathie zeigt und auf Anliegen der einzelnen Teammitglieder eingeht. Somit fühlen sich die Teammitglieder gut verstanden sowie behandelt, was die Beziehung der Teammitglieder untereinander zugute kommt. Außerdem ist es wichtig das sich die einzelnen Teammitglieder Feedback geben. Dabei sollten persönliche Beurteilungen zu anderen Teammitglieder unterlassen werden, um keine unnötigen persönlichen Konflikte zwischen den Teammitgliedern zu entfachen. Jedoch ist eine konstruktive Kritik eine gute Sache, da die Teammitglieder sich dadurch mehr geschätzt fühlen und insgesamt zufriedener mit dem Team sind.$^1$ Das steigert dann dann auch die Motivation im Team zu arbeiten und bietet eine gute Grundlage für eine positive Arbeitsatmosphäre im Team.
	\\ \\
	…kann mithilfe des Wasserfallmodells geschehen. Hierbei werden in der Analyse die Kundenanforderungen gesammelt, im Design das Produkt geplant, in der Programmierung das Produkt geschrieben und am Ende getestet. Nach etwa dreimaligem Wiederholen dieses Modells wird in der Regel ein gutes Ergebnis erzielt. Ein weiterer großer Vorteil des Wasserfallmodells ist die hohe Planungssicherheit, auch bei großen Projekten, in denen die Planung maßgeblich zum Erfolg beiträgt. Man darf aber nicht vergessen, dass das Wasserfallmodell eine unflexible Variante der Projektplanung ist und deswegen eher weniger in Projekten mit unabsehbaren Faktoren verwendet werden sollte.$^2$
	\\ \\
	…braucht Regeln, damit die einzelnen Teammitglieder wissen, wie sie sich zu verhalten haben. Ebenso wichtig ist eine Projektplanung, in der zum Beispiel das Projektziel und die Arbeitspakete definiert und zeitlich begrenzt wurden. Nicht zu unterschätzen ist die Teamgröße, denn ein zu großes Team ist schwer zu steuern, da alle Teammitglieder sich mit ihren Ideen im Projekt einbringen wollen, was zur Abweichung zum Projektziel führen könnte.$^3$ Außerdem steuern regeln den Arbeitsablauf im Team und sorgen für Struktur in der Zusammenarbeit der Teammitglieder. Jedoch sollten Regeln nur die für das Projektteam wichtigsten Aspekte regeln, wie zum Beispiel die Verhaltensweise Einzelner, um Teammitglieder nicht zu stark einzuschränken. Werden Teammitglieder durch Regeln zu stark eingeschränkt, kann dies dazu führen, dass sie bei Allem auf Regeln achten müssen und nicht richtig ihre eigenen Ideen im Projekt einbinden können, was auf Dauer demotivierend wirkt und möglicherweise einzelne Teammitglieder sich nicht mehr mit dem Projekt identifizieren können.
	\\ \\
	…hat gute Teamleiter, die in erster Hinsicht in die Situation der Teammitglieder einfühlen und mit ihnen umgehen können. Zudem sollten sie ihre Teammitglieder anführen, motivieren und fördern anstatt sie zurückzulassen. Gute Teamleiter machen klare Zielvorgaben und treffen Entscheidungen für das Team. Außerdem erkennen gute Teamleiter Konflikte im Team und schlichten diese.$^4$ Des Weiteren sollten Gute Teamleiter fair zu ihren Teammitgliedern sein und sie nicht herabwürdigen. Ein solches Verhalten von Teamleitern führt viel eher zu mehr Konflikten im Team und ist eine schlechte Grundlage für die Zusammenarbeit des Teams. Gute Teamleiter motivieren ihre Teammitglieder.$^5$ Vor allem dann, wenn sie demotiviert sind, um die Arbeitsatmosphäre zu stärken.
\\ \\
\section*{Quellen}
	1: grape, https://www.grape.io/de/blog/kommunikation-im-team-pflegen-und-verbessern, Stand: 20.01.2022\\
	2: pinuts, https://www.pinuts.de/projektmanagement-wasserfall-modell-gegen-agiles-arbeiten, Stand: 20.01.2022\\
	3: Computerwoche, https://www.computerwoche.de/a/10-tipps-fuer-erfolgreiche-it-projekt,2489690, Stand: 16.01.2022\\
	4: SPENDIT AG, https://www.spendit.de/magazin/tipps-fuer-teamleiter-so-fuehren-sie-ihr-team-noch-einfacher/, Stand: 16.01.2022\\
	5: karrierebibel, https://karrierebibel.de/teamleiter/, Stand: 20.01.2022
	
	
 
\end{document}