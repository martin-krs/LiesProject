\documentclass[12pt]{scrartcl}
 
\usepackage{graphicx}
\usepackage[english,german]{babel}
 
\begin{document}

\begin{titlepage}
	\centering
	{\scshape\LARGE Ostfalia Hochschule \par}
	\vspace{1cm}
	{\scshape\Large EKdI-Projekt\par}
	\vspace{1.5cm}
	{\huge\bfseries Pflichtenheft\par}
	\vspace{2cm}
	{\Large\itshape Martin Krause, Tom Strunz, Patricia Weber, Kristin Wachenschwan\par}
	\vfill
	supervised by\par
	Prof. \textsc{Jensen}

	\vfill

% Bottom of the page
	{\large \today\par}
\end{titlepage}


\tableofcontents
\newpage


\section{Zweck und Ziel}
Es soll ein Programm in Java >= 11, mit Open JDK, entwickelt werden, welches mit dem Aufrufen des \glqq Lies.java\grqq Java-Programms den Inhalt einer beliebigen Webseite vorlesen soll. Dabei muss beachtet werden das die nicht dargestellten Bestandteile, zum Beispiel Tags und Metadaten, nicht vorgelesen werden sollen. Dieses Programm soll entwickelt werden um zum Beispiel Personen mit Sehbehinderung Webseiten hinter URLs inhaltlich zeigen zu können. Dieses Pflichtenheft basiert auf dem dazugehörigem Lastenheft und beinhaltet alle Zwecke und Ziele des Lastenhefts außer sie werden ausgeschlossen.

\section{Abgrenzung}
Das Programm soll kein Graphic User Interface (GUI) enthalten und dem Programm wird es ebenfalls fast nur möglich sein Deutsche Webseiten vorzulesen, aufgrund von den auf Deutsch eingelesenen Silben. Das Proramm wird unter Windows geschrieben und getestet.

\section{Begriffe}


\section{Soll-Stand}


\subsection{Akteure}

\begin{itemize}
	\item LA1 - Hauptprogramm (Lies)
	\item LA2 - WebpageReader
	\item LA3 - AudioPlayer
\end{itemize}

\subsection{Funktionen}

\begin{itemize}
	\item Übergebe beim Start die URL
\begin{itemize}
	\item java Lies www.ostfalia.de
\end{itemize}
\end{itemize}

\begin{itemize}
	\item Vorlesen starten, stoppen, weiterspielen oder beenden
\begin{itemize}
	\item bla
\end{itemize}
\end{itemize}

\begin{itemize}
	\item Text aus Webseite ziehen und Steuerzeichen herausfiltern
\begin{itemize}
	\item bla
\end{itemize}
\end{itemize}

\begin{itemize}
	\item Text in Silben aufteilen
\begin{itemize}
	\item bla
\end{itemize}
\end{itemize}

\begin{itemize}
	\item Silben vorlesen lassen
\begin{itemize}
	\item bla
\end{itemize}
\end{itemize}

\subsection{Daten}
Die ersten Daten mit dem das Programm konfrontiert wird ist die übergebene URL. Nachdem mithilfe dieser
die Website geöffnet wurde, ist der gesamte HTML-Code in einem String gespeichert, der daraufhin bearbeitet
und in kleinere Strings mit ausschließlich Inhalt aufgeteilt wird. Außerdem benötigt das Programm sämtliche
wav-Dateien mit den gesprochenen Silben. Im Falle einer falsch übergebenen URL wird das Programm keinen nützlichen Output liefern. Sollte die URL ganz vergessen werden, wird der Nutzer dazu aufgefordert eine URL zu übergeben und das Programm beendet sich.


\subsection{Regeln}


\subsection{Nichtfunktionale Anforderungen}


\section{Dokumentenhistorie}

\begin{enumerate}
	\item | Martin Krause | Ersterstellung | 06.12.21
	\item | Kristin Altmann | Bearbeitung Punkt 1,2  | 06.12.21
	\item | Martin Krause | Deckblatt, Inhaltsverzeichnis, Funktionen bearbeitet | 13.12.21
\end{enumerate}
 
\end{document}
