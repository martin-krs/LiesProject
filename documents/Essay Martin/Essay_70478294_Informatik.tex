\documentclass[12pt]{scrartcl}
 
\usepackage{graphicx}
\usepackage{hyperref}
\usepackage[english,german]{babel}
 
\begin{document}

\begin{titlepage}
	\centering
	{\scshape\LARGE Ostfalia Hochschule \par}
	\vspace{1cm}
	{\scshape\Large EKdI-Projekt\par}
	\vspace{1.5cm}
	{\huge\bfseries Essay\par}
	\vspace{2cm}
	{\Large\itshape Martin Krause 70478294\par}
	\vfill
	supervised by\par
	Prof. \textsc{Jensen}

	\vfill

% Bottom of the page
	{\large \today\par}
\end{titlepage}



\section{Informatik ist ...}

...\glqq die Wissenschaft von den elektronischen Datenverarbeitungsanlagen und den Grundlagen ihrer Anwendung\grqq $^1$, wenn man im Duden nachschlägt, oder \glqq die Wissenschaft von der systematischen Darstellung, Speicherung, Verarbeitung und Übertragung von Informationen, wobei besonders die automatische Verarbeitung mit Digitalrechnern betrachtet wird\grqq $^2$, falls man bei Wikipedia nachguckt.
\newline
\newline
Informatik ist mittlerweile eine sehr große Wissenschaft geworden und lässt sich eigentlich gar nicht durch so eine kurze Definition beschreiben. Viele denken immer, dass die Informatik immer nur mit Computern zu tun hat - was irgendwo auch stimmt - jedoch liegen die Anfänge der Informatik weit zurück. Schon vor tausenden Jahren haben die Menschen simple Rechenregeln bzw. Ablaufregeln erschaffen und damit die ersten Algorithmen kreiert. Man bedenke, dass der \glqq Euklidische Algorithmus\grqq  bereits um 300 vor Christus von Euklid beschrieben wurde, um den größten gemeinsamen Teiler zweier natürlicher Zahlen zu finden$^3$.
\newline
\newline
Seit dem wuchs das Themengebiet wa wir heute als Informatik bezeichnen würde und es wurden mehr Rechenregeln aufgestellt oder schon simple Geräte für einfache Rechensysteme verwendet wie die jedem bekannten Rechenschieber$^4$, damit man sich Arbeit sparen konnte. Ende des 19. Jahrhunderts hat schließlich Hermann Hollerith im Rahmen einer Volkszählung in den USA eine elektronische Zählmaschine entwickelt, die die Volkszählung mithilfe von stanzbaren Lochkarten vereinfachte, sodass diese bereits nach einem halben Jahr abgeschlossen werden konnte$^5$. Er gründete außerdem eine Firma, die später in die Firma IBM überging, welche quasi revolutionär für die Entwicklung der ersten Computer war.
\newline
\newline
Seit dem vergangenen Jahrhundert, hat sich die Informatik dann sehr weiterentwickelt. Aber zuerst ging das alles nur theoretisch von statten. Das Themengebiet der theoretischen Informatik wurde immer präsenter, bis es schließlich Alan Turing (1912-1954) während des zweiten Weltkrieges gelang den quasi ersten Computer zu entwickeln, der eigentlich dazu diente, die Enigma-Codes der deutschen Wehrmacht zu knacken. Diese Maschine galt zumindest als die erste, die turing-vollständig ist, das heißt, sie beschreibt eine universelle Programmierbarkeit.
\newline
\newline
Durch die ständige Entwicklung wurden dann auch recht schnell erste Computer gebaut, der größtenteils aus elektromagnetischen Bauteilen bestanden. Dazu beigetragen hat der deutsche Bauingenieur Konrad Zuse (1910-1955) (siehe \glqq Grundlagen der Informatik\grqq$^6$.
\newline
In den 1970er Jahren wurde bereits die erste Hochsprache unter den Programmiersprachen entwickelt, also abseits von Assembler Code, namens \glqq C\grqq , welche auch heute noch sehr oft für schnelle, eingebettete Systeme benutzt wird. Seit dem haben sich hunderte Programmiersprachen geformt, mit allen möglichen Paradigmen, wie z.B. funktional, oder objekt-orientiert$^6$.
\newline
\newline
Etwa im selben Zeitraum wurden dann auch Grafikkarten entwickelt, die anders als die Central Processing Unit eines Computers viel mehr Kerne besitzt und deshalb gerade die grafiklastigen Aufgaben besser bearbeiten kann, da viele Berechnungen parallel erfolgen können. Diese Technologie war natürlich auch für die Filmindustrie wertvoll und führte dazu, dass \glqq Star Wars\grqq  der erste Film war, in dem wirklich gute computergenerierte Grafikeffekte vorkommen konnten.
\newline
\newline
Man sieht also dass die Informatik einen riesigen Teil unseres Lebens ausmacht, der gar nicht mehr wegzudenken ist. Schließlich ist heutzutage in jeder Waschmaschine und jedem noch so unscheinbaren Gerät ein Prozessor enthalten und ohne Internet würde in der heutigen Welt natürlich gar nichts mehr gehen. Man kann sich nur sehr schlecht vorstellen was passieren würde, wenn das Internet für nur einen Tag zusammenbrechen würde, aber die Folgen wären fatal, allein in der Infrastruktur, oder in der medizinischen Versorgung, Elektrische Werke und viele andere unscheinbare Systeme würden wahrscheinlich einen Totalausfall erleiden.
\newline
\newline
Es gibt sehr viele Möglichkeiten sich mit Themen aus Bereichen der theoretischen, technischen oder praktischen Informatik auseinanderzusetzen, auch wenn vieles davon wahrscheinlich Logik oder nur Mathematik ist, aber ich persönlich sehe die Informatik als eine große Kunstform an.

\newpage

\section{Quellenverzeichnis}

[1] Duden, \url{https://www.duden.de/rechtschreibung/Informatik}, Stand: 18.01.2022.
\newline
[2] Wikipedia, \url{https://de.wikipedia.org/wiki/Informatik}, Stand: 18.01.2022.
\newline
[3] Wikipedia, \url{https://de.wikipedia.org/wiki/Euklidischer_Algorithmus}, Stand: 18.01.2022.
\newline
[4] Rechnerlexikon, \url{http://rechnerlexikon.de/artikel/Geschichte_des_Rechenschiebers}, Stand: 18.01.2022.
\newline
[5] Damals.de, \url{https://www.wissenschaft.de/zeitpunkte/8-januar-lochkarten-fuer-die-volkszaehlung/}, Stand: 19.01.2022.
\newline
[6] Herold, H.; Lurz, B.; Wohlrab, J.; Hopf, M.; \glqq Grundlagen der Informatik \grqq , 3. Auflage, Pearson-Verlag, 2017.

 
\end{document}
