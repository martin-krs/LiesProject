\documentclass[12pt]{scrartcl}
 
\usepackage{graphicx}
\usepackage[english,german]{babel}
 
\begin{document}

\begin{titlepage}
	\centering
	{\scshape\LARGE Ostfalia Hochschule \par}
	\vspace{1cm}
	{\scshape\Large EKdI-Projekt\par}
	\vspace{1.5cm}
	{\huge\bfseries Essay\par}
	\vspace{2cm}
	{\Large\itshape Martin Krause 70478294\par}
	\vfill
	supervised by\par
	Prof. \textsc{Jensen}

	\vfill

% Bottom of the page
	{\large \today\par}
\end{titlepage}



\section{Informatik ist ...}

...die Wissenschaft von den elektronischen Datenverarbeitungsanlagen und den Grundlagen ihrer Anwendung, wenn man im Duden nachschlägt, oder die Wissenschaft von der systematischen Darstellung, Speicherung, Verarbeitung und Übertragung von Informationen, wobei besonders die automatische Verarbeitung mit Digitalrechnern betrachtet wird, falls man bei Wikipedia nachguckt.
\newline
\newline
Informatik ist mittlerweile eine sehr große Wissenschaft geworden und lässt sich eigentlich gar nicht durch so eine kurze Definition beschreiben. Viele denken immer, dass die Informatik immer nur mit Computern zu tun hat - was irgendwo auch stimmt - jedoch liegen die Anfänge der Informatik weit zurück. Schon vor tausenden Jahren haben die Menschen simple Rechenregeln bzw. Ablaufregeln erschaffen und damit die ersten Algorithmen kreiert. Man bedenke, dass der \glqq Euklidische Algorithmus\grqq  bereits um 300 vor Christus von Euklid beschrieben wurde (siehe Wikipedia), um den größten gemeinsamen Teiler zweier natürlicher Zahlen zu finden.
\newline
\newline
Seit dem vergangenen Jahrhundert, hat sich die Informatik sehr entwickelt. Aber zuerst ging das alles nur theoretisch von statten. Das Themengebiet der theoretischen Informatik wurde immer präsenter, bis es schließlich Alan Turing (1912-1954) während des zweiten Weltkrieges gelang den quasi ersten Computer zu entwickeln, der eigentlich dazu diente, die Enigma-Codes der deutschen Wehrmacht zu knacken. Diese Maschine galt zumindest als die erste, die turin-vollständig ist, das heißt, sie beschreibt eine universelle Programmierbarkeit.
\newline
Durch die ständige Entwicklung wurden dann auch recht schnell erste Computer gebaut, der größtenteils aus elektromagnetischen Bauteilen bestanden. Dazu beigetragen hat der deutsche Bauingenieur Konrad Zuse (1910-1955) (siehe \glqq Grundlagen der Informatik\grqq  , Pearson-Verlag, Seite 34).
\newline
In den 1970er Jahren wurde bereits die erste Hochsprache unter den Programmiersprachen entwickelt, also abseits von Assembler Code, namens \glqq C\grqq , welche auch heute noch sehr oft für schnelle, eingebettete Systeme benutzt wird. Seit dem haben sich hunderte Programmiersprachen geformt, mit allen möglichen Paradigmen, wie z.B. funktional, oder objekt-orientiert (siehe \glqq Grundlagen der Informatik\grqq  , Pearson-Verlag, Seite 149).
\newline
\newline
Man sieht also dass die Informatik einen riesigen Teil unseres Lebens ausmacht, der gar nicht mehr wegzudenken ist. Schließlich ist heutzutage in jeder Waschmaschine und jedem noch so unscheinbaren Gerät ein Prozessor enthalten und ohne Internet würde in der heutigen Welt natürlich gar nichts mehr gehen.
\newline
Es gibt sehr viele Möglichkeiten in theoretischer, technischer und praktischer Informatik, aber ich persönlich sehe die Informatik als eine große Kunstform an.

\newpage

\section{Ein erfolgreiches IT-Projekt ...}

...ist gar nicht so einfach zu realisieren. Wenn ich mein theoretisches Wissen und meine eigene praktische Erfahrung eines Software-Team-Projektes von meiner Zeit an der Technischen Universität Braunschweig mit einfließen lasse, gehört viel dazu ein IT-Projekt erfolgreich abzuschließen.
\newline
\newline
Angefangen bei der Teambildung, die schon Schwierigkeiten machen kann, da eventuell einige Leute im Team weniger Erfahrung mit diesem Tool oder dieser Programmiersprache haben als andere, oder weil Streit ausbricht. Es braucht teilweise gute Teamleiter, die das gesamte Projekt steuern und unterstützen.
\newline
Nach dem Erhalt des Auftrages für eine Software muss erstmal viel geplant werden bevor es überhaupt mit der Programmierung losgeht, da sonst direkt am Anfang schlechter Code entsteht, der am Ende überhaupt nicht das tut, was er soll. Solche Teams arbeiten für gewöhnlich nach bestimmten Strategien, wie z.B. dem V-Modell, dem Wasserfall-Modell, oder Scrum. Sobald man die Planung abgeschlossen hat und sämtliche Diagramme entworfen hat wie Aktivitäts- oder Zustandsdiagramme, oder auch Klassendiagramme, kann man mit der Programmierung anfangen. Dabei sollte allerdings auf bestimmte Coding-Richtlinien geachtet werden für Zeicheneinrückung oder Variablenbenennung.
\newline
\newline
Ein ganz wichtiger Punkt eines solchen Projektes stellt auch das ausgiebige Testen dar. Das bedeutet man testet kleine Programmteile für sich und setzt diese Stücke immer mehr zusammen, sodass man in Integrationstests testet, wie und ob die Teile zusammenarbeiten bist hin zum kompletten Programm. Dabei testet man am besten durch Whitebox-Tests, bei denen man die Bedingungen und den Funktionsinhalt kennt, und durch Blackbox-Tests, bei denen man keinen Einblick in Code hat und nur Randbedingungen und besondere Fälle von Variablen testet.
\newline
\newline
Und wenn man nach all dem dann das vermeintlich fertige Produkt dem Kunden vorstellt, kann es immer noch sein, dass dieser einige Stellen nicht gut findet und daraufhin nochmal sämtliche Codestücke angepasst werden müssen. Deswegen ist es wichtig von vornherein gut verständlichen und gut wartbaren Code zu schreiben, denn eine Funktion oder Klasse neu zu schreiben ist in den meisten Fällen einfacher, als den alten Code zu verbessern.
\newline
\newline
Insgesamt sollte ein erfolgreiches IT-Projekt also ein gutes Team haben, mit guter Arbeitsmoral, kompetenten Teammitgliedern und einer guten Arbeitsstrategie, sodass durch eine gute, überlegte Planung auch funktionierender Code geschrieben werden kann, der möglichst schnell zu guten Testergebnissen führt und ein erfolgreiches Produkt liefert.
\newline
Wenn man all das beachtet, sollten absolut verheerende Fehler bzw. Software-Krisen eigentlich nicht mehr vorkommen können (siehe \glqq Grundlagen der Informatik\grqq  , Pearson-Verlag, Seite 534).

 
\end{document}
