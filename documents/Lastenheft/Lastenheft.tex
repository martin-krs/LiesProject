\documentclass[11pt]{scrartcl}
 
 
\begin{document}

\begin{center}
\textbf{Lastenheft}
\end{center}

\section{Zweck und Ziel}
Es soll ein Programm entwickelt werden, welches mit dem Aufrufen des „Lies.java“ Java-Programms den Inhalt einer beliebigen URL vorlesen soll. Dabei muss beachtet werden das die nicht dargestellten Bestandteile, zum Beispiel Tags und Metadaten, nicht vorgelesen werden sollen. Dieses Programm soll entwickelt werden um zum Beispiel Personen mit Sehbehinderung URLs inhaltlich „zeigen“ zu können

\section{Abgrenzung}
Das Programm soll kein Graphic User Interface (GUI) enthalten und dem Programm wird es ebenfalls fast nur möglich sein Deutsche URLs vorzulesen, aufgrund von den auf Deutsch eingelesenen Silben.

\section{Begriffe}


\section{Soll-Stand}


\subsection{Akteure}

\begin{itemize}
	\item Hauptprogramm (Lies)
	\item WebpageReader
	\item AudioPlayer
\end{itemize}

\subsection{Funktionen}

\begin{itemize}
	\item URL bei Programmstart übergeben
	\item Vorlesen starten
	\item Vorlesen stoppen und weiterspielen lassen
	\item Programm beenden
\end{itemize}

\subsection{Daten}

Die ersten Daten mit dem das Programm konfrontiert wird ist die übergebene URL. Nachdem mithilfe dieser
die Website geöffnet wurde, ist der gesamte HTML-Code in einem String gespeichert, der daraufhin bearbeitet
und in kleinere Strings mit ausschließlich Inhalt aufgeteilt wird. Außerdem benötigt das Programm sämtliche
wav-Dateien mit den gesprochenen Silben.

\subsection{Regeln}


\subsection{Nichtfunktionale Anforderungen}


\section{Dokumentenhistorie}

\begin{enumerate}
	\item | Martin Krause | Ersterstellung | 01.11.21
	\item | Kristin Altmann | Bearbeitung (1. & 2. Punkt) | 06.11.21
\end{enumerate}
 
\end{document}
