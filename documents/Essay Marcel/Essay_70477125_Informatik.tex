\documentclass[12pt]{scrartcl}
 
\usepackage{graphicx}
\usepackage[english,german]{babel}
\usepackage{setspace}
\usepackage{url}
 
\begin{document}

\begin{titlepage}
	\centering
	{\scshape\LARGE Ostfalia Hochschule \par}
	\vspace{1cm}
	{\scshape\Large EKdI-Projekt\par}
	\vspace{1.5cm}
	{\huge\bfseries Essay\par}
	\vspace{2cm}
	{\Large\itshape Marcel Pulst, 70477125\par}
	\vfill
	supervised by\par
	Prof. \textsc{Jensen}

	\vfill

% Bottom of the page
	{\large \today\par}
\end{titlepage}



\section*{Informatik ist ...}
\onehalfspacing
	…eine Wissenschaft, die von der systematischen Verarbeitung von Informationen handelt. Hierbei werden Informationen besonders automatisch mit Computern verarbeitet. Die Informatik untersucht Methoden, um Informationen zu verarbeiten und diese Methoden in verschiedensten Bereichen zu nutzen. Die Informatik teilt sich in theoretischer und praktischer Informatik auf. Dabei werden bei der theoretischen Informatik Algorithmen und Syntax für Programmiersprachen entwickelt, Automaten auf Basis mathematischer Modelle erstellt und Schaltungen sowie komplexe Schaltkreise untersucht. Generell wird bei der theoretischen Informatik viel Mathematik und auch durchaus schwierige Mathematik verwendet. Bei der praktischen Informatik werden die in der theoretischen Informatik entwickelten Methoden eingesetzt. Hierbei wird eher weniger Mathematik verwendet, um etwas neues, wie zum Beispiel einen neuen Algorithmus, zu erschaffen.$^1$
	\\ \\
	…die Informationsübertragung von einem Sender zu einem Empfänger. Hierbei wird über ein Informationskanal eine Nachricht mit einem Informationsgehalt versendet, wobei ein Datenverlust entstehen kann (z. B. über WLAN). Dabei können analoge Signal ins Digitale umgewandelt werden. Informationen oder auch Daten werden zwischen zwei bzw. mehreren Rechnern mithilfe von zweckmäßigen Übertragungsmethoden übermittelt werden.$^2$
	\\ \\
	…das Programmieren von Programmen, um Aufgaben bzw. Probleme zu lösen. Hierbei hilft die Programmiersprache, die für Menschen lesbar ist, ab. Die Programmiersprache wird am Ende zu Maschinencode umgewandelt, damit auch der Computer den Code versteht und das geschrieben Programm umsetzt. Die Programmiersprache ist außerdem ein essentieller Teil vom Bild , dass wir heutzutage von der Informatik haben.$^3$ Das Programmieren findet oftmals in Gruppen statt, um große Programme zu erschaffen. Damit dies einfacher gelingt ist ein sauberer Programmierstil wichtig, damit andere Programmierer dienen Programmcode auch lesen können. Beim Programmieren hat der Programmierer außerdem auch auf die Syntax seiner bzw. ihrer Programmiersprache zu achten. Außerdem ist es noch sehr wichtig, dass wenn in einer Gruppe ein Programmierer eine oder mehrere Dateien verändert hat, dass dieser Programmierer allen nötigen Gruppenmitgliedern mitteilt, was genau verändert wurde, damit die Gruppenmitglieder die Veränderungen nachvollziehen und mit diesen auch weiterarbeiten können. Gruppenmitglieder, die am Programm weiterarbeiten und ohne über die Veränderungen in Kenntnis gesetzt wurden, könnten plötzliche Fehlermeldungen bekommen, da sie fälschlicherweise dachten, sie hätten am alten Programm weitergearbeitet und Dinge geändert oder hinzugefügt, die mit dem aktualisierten Programm aber nicht funktionieren.
	\\ \\
	…ein Zusammenspiel von Hardware und Software. Dabei spielt das Betriebssystem eine entscheidende Rolle, denn es ist die Schnittstelle von Hardware und Software und ist im ROM unlöschbar gespeichert.$^4$ Mit dem Betriebssystem können Informationen aus der Hardware gelesen und in Software, die Menschen verstehen können, weitergeleitet werden. Umgekehrt kann das Betriebssystem Informationen von der Software in die Hardware übertragen.\\
	Die Hardware beziehungsweise die Komponenten des Computers arbeiten mit elektrischen Impulsen, welche Daten darstellen. Die Hardware bezeichnet jene Komponenten des Computers, die physische Bestandteile dessen sind. Die Software hingegen bezeichnet jene Komponenten des Computers, die nicht physisch sind. Darunter fallen alle Programme, die Daten verarbeiten und für den Betrieb des Rechners verantwortlich sind. Die durch diese Programme erstellten Daten fallen ebenfalls unter dem Begriff der Software.$^5$
\\ \\
\section*{Quellen}
	1: Gabler Wirtschaftslexikon, https://wirtschaftslexikon.gabler.de/definition/informatik-3849, Stand: 16.01.2022\\
	2: SoftGuide, https://www.softguide.de/software-tipps/datenuebertragung, Stand: 19.01.2022\\
	3: THE DIGITAL TALENTS,https://www.thedigitaltalents.com/programmieren-informatik-unterschied, Stand: 19.01.2022\\
	4: Elektronik Kompendium, https://www.elektronik-kompendium.de/sites/com/1401041.htm, Stand: 16.01.2022\\
	5: WINTotal, https://www.wintotal.de/hardware-software/, Stand: 19.01.2022
\end{document}