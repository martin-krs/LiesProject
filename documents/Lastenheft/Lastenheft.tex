\documentclass[11pt]{scrartcl}
 
 
\begin{document}

\begin{center}
\textbf{Lastenheft}
\end{center}

\section{Zweck und Ziel}


\section{Abgrenzung}


\section{Begriffe}


\section{Soll-Stand}


\subsection{Akteure}

\begin{itemize}
	\item Hauptprogramm (Lies)
	\item WebpageReader
	\item AudioPlayer
\end{itemize}

\subsection{Funktionen}

\begin{itemize}
	\item URL bei Programmstart übergeben
	\item Vorlesen starten
	\item Vorlesen stoppen und weiterspielen lassen
	\item Programm beenden
\end{itemize}

\subsection{Daten}

Die ersten Daten mit dem das Programm konfrontiert wird ist die übergebene URL. Nachdem mithilfe dieser
die Website geöffnet wurde, ist der gesamte HTML-Code in einem String gespeichert, der daraufhin bearbeitet
und in kleinere Strings mit ausschließlich Inhalt aufgeteilt wird. Außerdem benötigt das Programm sämtliche
wav-Dateien mit den gesprochenen Silben.

\subsection{Regeln}


\subsection{Nichtfunktionale Anforderungen}


\section{Dokumentenhistorie}

\begin{enumerate}
	\item | Martin Krause | Ersterstellung | 01.11.21
\end{enumerate}
 
\end{document}