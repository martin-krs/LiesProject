\documentclass[12pt]{scrartcl}
 
\usepackage{graphicx}
\usepackage[english,german]{babel}
 
\begin{document}

\begin{titlepage}
	\centering
	{\scshape\LARGE Ostfalia Hochschule \par}
	\vspace{1cm}
	{\scshape\Large EKdI-Projekt\par}
	\vspace{1.5cm}
	{\huge\bfseries Essay\par}
	\vspace{2cm}
	{\Large\itshape Martin Krause 70478294\par}
	\vfill
	supervised by\par
	Prof. \textsc{Jensen}

	\vfill

% Bottom of the page
	{\large \today\par}
\end{titlepage}



\section{Informatik ist ...}
...die Wissenschaft von den elektronischen Datenverarbeitungsanlagen und den Grundlagen ihrer Anwendung, wenn man im Duden nachschlägt, oder die Wissenschaft von der systematischen Darstellung, Speicherung, Verarbeitung und Übertragung von Informationen, wobei besonders die automatische Verarbeitung mit Digitalrechnern betrachtet wird, falls man bei Wikipedia nachguckt.
\newline
\newline
Informatik ist mittlerweile eine sehr große Wissenschaft geworden und lässt sich eigentlich gar nicht durch so eine kurze Definition beschreiben. Viele denken immer, dass die Informatik immer nur mit Computern zu tun hat - was irgendwo auch stimmt - jedoch liegen die Anfänge der Informatik weit zurück. Schon vor tausenden Jahren haben die Menschen simple Rechenregeln bzw. Ablaufregeln erschaffen und damit die ersten Algorithmen kreiert. Man bedenke, dass der \glqq Euklidische Algorithmus\grqq  bereits um 300 vor Christus von Euklid beschrieben wurde (siehe Wikipedia), um den größten gemeinsamen Teiler zweier natürlicher Zahlen zu finden.
\newline
\newline
Seit dem vergangenen Jahrhundert, hat sich die Informatik sehr entwickelt. Aber zuerst ging das alles nur theoretisch von statten. Das Themengebiet der theoretischen Informatik wurde immer präsenter, bis es schließlich Alan Turing während des zweiten Weltkrieges gelang den quasi ersten Computer zu entwickeln, der eigentlich dazu diente, die Enigma-Codes der deutschen Wehrmacht zu knacken. Diese Maschine galt zumindest als die erste, die turin-vollständig ist, das heißt, sie beschreibt eine universelle Programmierbarkeit.
\newline
Durch die ständige Entwicklung wurde dann auch recht schnell erste Computer gebaut, der größtenteils aus elektromagnetischen Bauteilen bestand. Dazu beigetragen hat der deutsche Bauingenieur Konrad Zuse (1910-1955) (siehe \glqq Grundlagen der Informatik\grqq  , Pearson, Seite 34).
\newline
In den 1970er Jahren wurde bereits die erste Hochsprache unter den Programmiersprachen entwickelt, also abseits von Assembler Code, namens \glqq c\grqq , welche auch heute noch sehr für schnelle, eingebettete Systeme benutzt wird. Seit dem haben sich hunderte Programmiersprachen geformt, mit allen möglichen Paradigmen, wie z.B. funktional, oder objekt-orientiert (siehe \glqq Grundlagen der Informatik\grqq  , Pearson, Seite 149).
\newline
\newline
Man sieht also dass die Informatik ein riesigen Teil unseres Lebens ausmacht, der gar nicht mehr wegzudenken ist. Schließlich ist heutzutage in jeder Waschmaschine und jedem noch so unscheinbaren Gerät ein Prozessor enthalten und ohne Internet würde in der heutigen Welt natürlich gar nichts mehr gehen.
\newline
Es gibt sehr viele Möglichkeiten in theoretischer, technischer und praktischer Informatik, aber ich persönlich sehe die Informatik als eine große Kunstform an.


\newpage

\section{Ein erfolgreiches IT-Projekt ...}



 
\end{document}
